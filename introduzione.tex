\section{Introduzione}
L'heap è una regione di memoria assegnata ad un processo, per dati che la cui esistenza o dimensione non è conosciuta a tempo di compilazione. A differenza dello stack, la vita di un'allocazione non dipende dalla procedura o dallo stack frame corrente. Questa memoria, infatti, è globale, quindi può essere acceduta e modificata da qualunque parte del programma. Si accede alla regione di memoria allocata in maniera indiretta, per mezzo di un puntatore. 

In linguaggio C si può allocare e deallocare memoria in heap tramite due funzioni, offerte dalla libreria standard \verb+stdlib.h+:
\begin{lstlisting}[style=CStyle]
	void* malloc(size_t size)
\end{lstlisting}
che alloca in heap uno spazio di dimensione \verb+size+, e
\begin{lstlisting}[style=CStyle]
	void free(void* p)
\end{lstlisting}
che dealloca dall'heap la zona di memoria puntata da \verb+p+, precedentemente allocata da \verb+malloc+.

Esistono una serie di regole, di cui è responsabile il programmatore, che se rispettate consentono di evitare bug:
\begin{enumerate}
\item liberare una zona di memoria con \verb+free+ ottenuta da una \verb+malloc+
\item non usare \verb+free+ su una zona di memoria più di una volta
\item assicurarsi di non sovrascrivere zone di memoria che eccedono la memoria richiesta dalla \verb+malloc+, per evitare \emph{heap overflow}.
\end{enumerate}